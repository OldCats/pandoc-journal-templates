% Options for packages loaded elsewhere
\PassOptionsToPackage{unicode}{hyperref}
\PassOptionsToPackage{hyphens}{url}
%
\documentclass[
  20pt
]{article}

\usepackage{layout}
\usepackage{authblk}
\usepackage[symbol,flushmargin]{footmisc}
\renewcommand{\thefootnote}{\fnsymbol{footnote}}
\renewcommand{\footnoterule}{%
  \kern 2em  % footnote 水平線上空間
  \hrule width \textwidth height 0.5pt % footnote 水平線寬
  \kern 0.5em % footnote 水平線下空間
}

\usepackage{ifthen}
% \usepackage[256, showframe]{geometry}
\usepackage[top=2.5cm, bottom=2.5cm]{geometry}
\usepackage{amsmath,amssymb}
\usepackage{multicol}
\setlength{\columnsep}{1.2cm}
\usepackage{graphicx}
\usepackage{lmodern}
\usepackage{iftex}

\usepackage{indentfirst}
\usepackage{fancyhdr}
\setlength{\parskip}{1em}
\usepackage{titlesec}

% \titlespacing{command}{left spacing}{before spacing}{after spacing}[right]
\titlespacing\section{0pt}{1pt}{0pt plus 2pt minus 2pt}
\titlespacing\subsection{0pt}{1pt plus 4pt minus 2pt}{0pt plus 2pt minus 2pt}
\titlespacing\subsubsection{0pt}{1pt plus 4pt minus 2pt}{0pt plus 2pt minus 2pt}

\geometry{a4paper, scale=0.75}

%------------header/footer setting-----------%

\pagestyle{fancy}{}
\fancyhf{}
\setcounter{page}{22}

\fancyhead[L] % 頁首左邊
{\ifthenelse{\value{page}=22} % 第1頁
    {\footnotesize \textbf{Journal of Acute Medicine} 11(1): 22-27, 2021\\[0.2em]
    \textbf{DOI}:10.6705/j.jacme.202103\_11(1).0004\\[0.2em]
    \textbf{Brief Report}
    }
    {\ifthenelse{\isodd{\value{page}}}
        {} % 其他奇數頁
        {Chiu et al.}} % 其他偶數頁
}

\fancyhead[R] % 頁首右邊
{\ifthenelse{\value{page}=22} % 第1頁
    {\includegraphics[width=1.5cm]{ap.jpg}
     \includegraphics[width=1.5cm]{jacme.jpg}
     \includegraphics[width=1.5cm]{doi-11-1-04.jpg}}
    {\ifthenelse{\isodd{\value{page}}}
        {\footnotesize Trauma Field Triage Training of EMT} % 其他奇數頁
        {}} % 其他偶數頁
}

\fancyfoot[L] % 左邊頁尾
{\ifthenelse{\isodd{\value{page}}}
    {}
    {\thepage\hspace{5pt}\footnotesize Journal of Acute Medicine 11(1) 2021}}

\fancyfoot[R] % 右邊頁尾
{\ifthenelse{\isodd{\value{page}}}
    {{\footnotesize Journal of Acute Medicine 11(1) 2021}\hspace{5pt}\thepage}
    {}}


\renewcommand{\headrulewidth}{0pt}

%-------------------------------------------%


%------------footnote without marker--------%
\newcommand\blfootnote[1]{%
  \begingroup
  \renewcommand\thefootnote{}\footnote{#1}%
  \addtocounter{footnote}{-1}%
  \endgroup
}
%-------------------------------------------%

% luatex or xetex setting
  \usepackage{unicode-math}
  \defaultfontfeatures{Scale=MatchLowercase}
  \defaultfontfeatures[\rmfamily]{Ligatures=TeX,Scale=1}


% \makeatletter

% \setlength{\parskip}{6pt plus 2pt minus 1pt}}
\usepackage{hyperref} %for citing reference

% \urlstyle{same} % disable monospaced font for URLs
\usepackage{longtable,booktabs,array}
\usepackage{calc} % for calculating minipage widths
% Correct order of tables after \paragraph or \subparagraph
\usepackage{etoolbox}
\makeatletter
\patchcmd\longtable{\par}{\if@noskipsec\mbox{}\fi\par}{}{}
\makeatother
% Allow footnotes in longtable head/foot
\IfFileExists{footnotehyper.sty}{\usepackage{footnotehyper}}{\usepackage{footnote}}
\makesavenoteenv{longtable}
\setlength{\emergencystretch}{3em} % prevent overfull lines
\providecommand{\tightlist}{%
  \setlength{\itemsep}{0pt}\setlength{\parskip}{0pt}}
\setcounter{secnumdepth}{-\maxdimen} % remove section numbering
\ifLuaTeX
  \usepackage{selnolig}  % disable illegal ligatures
\fi
\newlength{\cslhangindent}
\setlength{\cslhangindent}{1.5em}
\newlength{\csllabelwidth}
\setlength{\csllabelwidth}{3em}
\newenvironment{CSLReferences}[2] % #1 hanging-ident, #2 entry spacing
 {% don't indent paragraphs
  \setlength{\parindent}{0pt}
  % turn on hanging indent if param 1 is 1
  \ifodd #1 \everypar{\setlength{\hangindent}{\cslhangindent}}\ignorespaces\fi
  % set entry spacing
  \ifnum #2 > 0
  \setlength{\parskip}{#2\baselineskip}
  \fi
 }%
 {}
\usepackage{calc}
\newcommand{\CSLBlock}[1]{#1\hfill\break}
\newcommand{\CSLLeftMargin}[1]{\parbox[t]{\csllabelwidth}{#1}}
\newcommand{\CSLRightInline}[1]{\parbox[t]{\linewidth - \csllabelwidth}{#1}\break}
\newcommand{\CSLIndent}[1]{\hspace{\cslhangindent}#1}




\makeatletter 

% maketitle and author
\renewcommand{\maketitle}{\bgroup\setlength{\parindent}{0pt}
\begin{flushleft}
    \begin{minipage}{14cm}
        \textbf{\Large\@title}
    \end{minipage}

    \vspace*{3pt}
        \@author
\end{flushleft}\egroup
}



\renewcommand*{\Authfont}{\raggedright\sffamily\small}
\renewcommand*{\Affilfont}{\raggedright\rmfamily\scriptsize} % 修改機構名稱的字體與大小
\renewcommand\Authands{, } % 去掉 and 前的逗號
\date{} % 去掉日期





\begin{document}

\footnotetext[0]{Received: May 15, 2019; Revised: January 6, 2020 (3rd); Accepted: February 11, 2020}

$if(title)$
\title{$title$}
$endif$

$if(author)$
  $for(surname)$
    \author[2]{$surname$ $given-names$}
  $endfor$
$endif$

$for(author)$\author[$for(author.affiliation)$$author.affiliation$$sep$,$endfor$$if(author.corresp)$,\dagger$endif$]{$author.given-names$ $author.surname$$if(author.footnote)$\footnote[0]{\textsuperscript{\ast}$author.footnote$}$endif$}$endfor$

\affil[1]{Department of Emergency Medicine,
        Far Eastern Memorial Hospital, 
        New Taipei City, 
        Taiwan}
\affil[2]{Department of Emergency Medicine, 
        National Taiwan University Hospital, 
        Taipei, 
        Taiwan}
\affil[3]{National Fire Agency, 
        Ministry of the Interior, 
          Taiwan}
\affil[4]{Taipei City Fire Department, 
        Taipei, 
        Taiwan}
\affil[5]{Department of Emergency Medicine, 
        National Taiwan University Hospital Yun-Lin Branch, 
        Yunlin, 
        Taiwan}
        

\vspace*{5ex}

% \blfootnote{\textsuperscript{\dagger}These authors contributed equally to this work.}   

\maketitle
\thispagestyle{fancy} % render Title, author and affiliation

\vspace{1ex}

% Abstracts and keywords
\begin{minipage}{15cm}
    \vspace{5ex}
    



    {\parindent 2em Injury is a leading cause of death among young adults. An accurately
implemented filed triage scheme (FTS) by emergency medical technicians
(EMTs) is the first step for delivering right patients to the right
hospital. However, the training effect of FTS on EMTs with different
levels and backgrounds has scarcely been reported. We evaluated training
effects of FTS among EMTs in Taipei. Standard FTS contains physiologic
status, anatomical sites of injury, and mechanism of injury criteria.
The intervention was a 30-minute lecture and pre-and-post tests, each
containing five questions about trauma severity judgment (i.e.,
mechanism of injury {[}2 questions{]}, anatomic sites of injury {[}2
questions{]}, and physiological status {[}1 question{]}). The change in
EMT accuracy was measured before and after training. Subgroup analyses
were performed across EMTs with different levels and seniorities. From
September 1, 2015 to March 31, 2016, 821 EMTs were enrolled, including
740 EMT-intermediates and 81 paramedics. Overall, EMT accuracy improved
after the intervention in the intermediate (73.2\% vs.~85.5\%, p
\textless{} 0.05) and paramedic (76.0\% vs.~85.7\%, p \textless{} 0.01)
groups. All trainees showed improvements in physiology and mechanism
criteria, but paramedics showed decreased accuracy in anatomic criteria.
The subgroup analysis showed that accuracy positively associated with
prehospital care experience for major trauma cases 1 year before the
training course, and the anatomical criterion accuracy was adversely
associated with paramedic seniority. Field triage training can improve
EMT accuracy for FTS. The anatomical aspect is more diffi cult to
improve and should be emphasized in FTS training courses.}

    \vspace{1em}
    \textbf{Key words:} \textit{emergency medical service (EMS), emergency
medical technician (EMT), education and training, filed triage scheme
(FTS), trauma}

\end{minipage}

\vspace{2em}

\setlength{\parindent}{2em}

\begin{multicols}{2}
\hypertarget{introduction}{%
\section{Introduction}\label{introduction}}

{Injury is the ninth leading cause of death in the world, and the leading
cause of death in young people.\textsuperscript{{[}1{]}} Rapid
transporting a severely-injured patient to an appropriate trauma center
reduces his/her risk of mortality by 25\%\textsuperscript{{[}2{]}}
because prolonged prehospital time is associated with worsen
outcome,\textsuperscript{{[}3{]}} and that relies on accurate field
triage decisions by emergency medical technicians (EMTs). In many
countries, emergency medical service (EMS) systems develop field triage
scheme (FTS) in reference to the US FTS.\textsuperscript{{[}4{]}}
Although the US FTS guides the destination facility choice by a stepwise
evaluation of physiologic status (P), anatomical injury (A), injury
mechanism (M), and special consideration of the injured patient, EMTs'
responses to P, A, and M criteria may differ.\textsuperscript{{[}2{]}}

Currently, no study has investigated field triage training of EMTs.
Here, we evaluated whether a single training program improves field
triage accuracy and learning outcomes among FTS sub-aspects.}

\hypertarget{methods}{%
\section{Methods}\label{methods}}

\hypertarget{study-design-and-setting}{%
\subsection{Study Design and Setting}\label{study-design-and-setting}}

This is a before-and-after study. We held seven field triage trainings
for Taipei EMTs from September 1, 2015 to March 31, 2016 for all
intermediate and paramedic EMTs in Taipei City. Taipei City is a
metropolitan area with 2.65 million registered individuals over 272
km\textsuperscript{2} , including about 3.00 million inflow daytime
workers. There are 973 intermediates and 89 paramedics in Taipei City
and up to 49,000 annual trauma-related missions.

\hypertarget{irb-approval}{%
\subsection{IRB Approval}\label{irb-approval}}

The study protocol was approved by the Institutional Review Board of the
National Taiwan University Hospital.

\hypertarget{exposure-definition}{%
\subsection{Exposure Definition}\label{exposure-definition}}

Participants were exposed to a 30-minute lecture and underwent pre- and
post-tests. The pre-test and post-test were comprised of five
single-choice questions based on a scenario with trauma patient. Each
question regard to mechanism of injury (M; 2 questions), anatomic injury
sites (A; 2 questions), or physiological status (P; 1 question). A
medical director team in Taipei City designed the lecture and test
contents. Although the scenarios were different in preand post-test,
these two scenarios were designed by the medical director team with the
consensus of the degree of difficulty to guarantee comparability.

The seniority and major trauma experience among the paramedics were also
recorded to test the association with a training effect. The seniority
is comprised of ``EMT seniority'' (i.e., the period from the certificate
of EMT basic) and ``Paramedic seniority'' (i.e., the period from the
certification of EMT-P). The reason to divide the seniority is that we
think both the occupational years (EMT seniority with multi-tasking
duty) and expertise experience (paramedic seniority merely focusing on
the ambulance run) might differently affect their learning.

\hypertarget{outcomes}{%
\subsection{Outcomes}\label{outcomes}}

The primary outcome was the test accuracy comparison before and after
training. The secondary outcomes included EMT level and seniority.

\hypertarget{statistical-analysis}{%
\subsection{Statistical Analysis}\label{statistical-analysis}}

Data entered in Excel (Microsoft, Redmond, WA, USA) was processed and
analyzed by SAS software version 9.3 (SAS Institute, Cary, NC, USA).
Descriptive population statistics were given as counts, percentages, or
mean ± standard deviation. We used the McNemar and Mann--Whitney rank
sum tests for comparisons. A two-tailed p-value \textless{} 0.05 was
considered significant. We examined the correlation between accuracy and
(1) seniority, (2) experience of major trauma mission, and (3) major
trauma mission within 1 year using Pearson correlations.

\hypertarget{results}{%
\section{Results}\label{results}}

\end{multicols}

Fig. 1 \emph{shows} the participant algorithm. In total, 740
intermediates (76.1\% of all EMT-intermediates in Taipei) and 81
paramedics (91\% of all paramedics in Taipei) were included. Participant
characteristics are summarized in revised Table 1.

Fig. 1. Algorithm of this study.\\
EMT: emergency medical technician.

\begin{longtable}[]{@{}lll@{}}
\caption{Demographic data}\tabularnewline
\toprule
  & EMT-intermediate & EMT-paramedic \\
\midrule
\endfirsthead
\toprule
  & EMT-intermediate & EMT-paramedic \\
\midrule
\endhead
Number & 740 & 81 \\
Age, mean ± SD, y & 34.6 ± 7.2 & 34.7 ± 4.5 \\
Male (\%) & 702 (94.8) & 77 (95.1) \\
Seniority, mean ± SD, y & 5.6 ± 2.4 & 11.8 ± 5.1 \\
Major trauma mission within 1 year & --- & 61 \\
\bottomrule
\end{longtable}

\begin{multicols}{2}
The baseline field triage accuracy for all EMS providers was 73.51\% and
rose to 85.6\% after training (p \textless{} 0.05). Overall accuracy
improved after intervention in the intermediate (73.2\% vs.~85.5\%, p
\textless{} 0.05) and paramedic (76.0\% vs.~85.7\%, p \textless{} 0.01)
groups, which presented higher mechanism and physiological aspect
post-test accuracies, but not higher anatomical aspects. The trend of
anatomical question performance worsened in the paramedic group, but was
not statistically significant (82.0\% vs.~80.9\%, p = 0.87) in revised
Table 2. To determine the impact of seniority, we divided paramedics
into subgroups (Table 3). Significant mechanism aspect improvements were
observed in all subgroups. Physiological aspect performance was not
statistically significant but tended to improve. The anatomical aspect
accuracy non-significantly declined after training in the high seniority
subgroup. We analyzed the association between field triage and
experience in the paramedic group and found a significant correlation
between ``major trauma mission within 1 year'' and pretest/posttest
accuracy (p = 0.037/0.002).
\end{multicols}


\begin{longtable}[]{@{}lllllll@{}}
\caption{Improvement in accuracy after training}\tabularnewline
\toprule
& EMT-intermediate & EMT-paramedic & & & & \\
\midrule
\endfirsthead
\toprule
& EMT-intermediate & EMT-paramedic & & & & \\
\midrule
\endhead
& Pre-test & Post-test & p value & Pre-test & Post-test & p value \\
Overall, \% & 73.2 & 85.5 & \textless{} 0.05* & 76 & 85.7 & \textless{}
0.01* \\
Mechanism, \% & 68.4 & 92 & \textless{} 0.01* & 71.6 & 91.4 &
\textless{} 0.01* \\
Physiological, \% & 68.1 & 81.9 & 0.01* & 72.8 & 84 & 0.09 \\
Anatomical, \% & 78.2 & 83.4 & 0.2 & 82 & 80.9 & 0.87 \\
\bottomrule
\end{longtable}

\begin{longtable}[]{@{}
  >{\raggedright\arraybackslash}p{(\columnwidth - 8\tabcolsep) * \real{0.22}}
  >{\raggedright\arraybackslash}p{(\columnwidth - 8\tabcolsep) * \real{0.07}}
  >{\raggedright\arraybackslash}p{(\columnwidth - 8\tabcolsep) * \real{0.22}}
  >{\raggedright\arraybackslash}p{(\columnwidth - 8\tabcolsep) * \real{0.21}}
  >{\raggedright\arraybackslash}p{(\columnwidth - 8\tabcolsep) * \real{0.22}}@{}}
\caption{Effect of experience on accuracy in the paramedic
group}\tabularnewline
\toprule
Paramedic & n & Pre-test accuracy (\%) & Post-test

accuracy (\%) & McNemar

p value \\
\midrule
\endfirsthead
\toprule
Paramedic & n & Pre-test accuracy (\%) & Post-test

accuracy (\%) & McNemar

p value \\
\midrule
\endhead
Sex & & & & \\
Male & 77 & M = 72.7 & M = 91.6 & M, p \textless{}
0.001\textsuperscript{*} \\
& & P = 74.0 & P = 84.4 & P, p = 0.115 \\
& & A = 81.8 & A = 81.2 & A, p = 0.167 \\
EMT seniority (year) & & & & \\
≤ 12 & 39 & M = 65.4 & M = 92.3 & M, p \textless{}
0.001\textsuperscript{*} \\
& & P = 79.5

A = 80.8 & P = 92.3

A = 87.2 & P, p = 0.180

A, p = 0.039\textsuperscript{*} \\
\textgreater{} 12 & 39 & M = 76.9 & M = 93.6 & M, p \textless{}
0.001\textsuperscript{*} \\
& & P = 69.2 & P = 79.5 & P, p = 0.388 \\
& & A = 83.3 & A = 73.1 & A, p = 0.508 \\
Paramedic seniority (year) & & & & \\
≤ 5 & 39 & M = 67.9 & M = 92.3 & M, p \textless{}
0.001\textsuperscript{*} \\
& & P = 79.5 & P = 92.3 & P, p = 0.180 \\
& & A = 84.6 & A = 84.6 & A, p = 0.344 \\
\textgreater{} 5 & 38 & M = 75.0 & M = 93.4 & M, p \textless{}
0.001\textsuperscript{*} \\
& & P = 68.4 & P = 78.9 & P, p = 0.388 \\
& & A = 78.9 & A = 75.0 & A, p = 1.000 \\
Major trauma mission within 1 year & & & & \\
No & 20 & M = 62.5 & M = 85.0 & M, p \textless{}
0.001\textsuperscript{*} \\
& & P = 70.0 & P = 75.0 & P, p = 1.000 \\
& & A = 80.0 & A = 72.5 & A, p = 1.000 \\
Yes & 61 & M = 74.6 & M = 93.4 & M, p \textless{}
0.001\textsuperscript{*} \\
& & P = 73.8 & P = 86.9 & P, p = 0.057 \\
& & A = 82.8 & A = 83.6 & A, p = 0.302 \\
\bottomrule
\end{longtable}

\begin{multicols}{2}
\hypertarget{discussion}{%
\section{Discussion}\label{discussion}}

The single FTS training program significantly improved overall
prehospital field triage decision accuracy. Improvement was achieved
mainly in mechanical and physiological criteria. To our knowledge, this
is the first study to evaluate different FTS aspects among EMTs, and our
results are informative for FTS training courses. In our study, major
trauma within 1 year positively impacted field triage. This was
supported by David and Brachet's study, which found that more than 2
years of field experience plus past and recent trauma patient volume
associated with prehospital interval
reductions.\textsuperscript{{[}4{]}} However, the anatomical aspect
accuracy decreased in the paramedic group, especially for those with
higher seniority, which seems to thwart educational improvement. Most
senior paramedics incorrectly answered the post-test question that
pictured a javelin penetrating the calf (Fig 2), misleading to
over-triage this case to a trauma center, while a penetrating distal
limb injury does not meet trauma center transportation criteria.
However, there is a proviso, ``when in doubt, transport to a trauma
center''.\textsuperscript{{[}3{]}} Even though senior paramedics
incorrectly answered this question; this over-triage is acceptable in
clinical practice. The US-FTS evaluates injured patients step by step.
However, studies that have discussed the prehospital triage
decision-making model revealed that the information obtained before
contacting the patient (early visual cues from the scene, mechanism of
injury, and patient appearance) play more important roles than further
assessment (defi nite vital signs, non-obvious anatomic injury, and
detail history) in decision
making.;\textsuperscript{{[}5{]}}\textsuperscript{{[}6{]}} ``EMT gut
feeling'' refers to the ``EMS provider judgment'' criteria that was the
last criteria added into the ACS COT (American College of Surgeons
Committee on Trauma) decision scheme in 2006 and is the last triage
scale step. However, it is also the most commonly cited (40.0\%) and
sometimes the only used (21.4\%) criterion.5 The value of EMS provider
judgement criteria varies between sites, modestly increases overtriage,
and plays an important role in identifying high risk patients who are
missed by physiologic and anatomic
criteria.;\textsuperscript{{[}5{]}}\textsuperscript{{[}6{]}} In our
study, the post-test penetration question is sanguineous and makes the
trainees follow their gut feeling and triage the patient to the trauma
center rather than apply the anatomical step. Further training programs
should put more emphasis on avoiding EMT judgments that increase
overtriage rates and impact emergency care efficacy.

Fig. 2. One of the questions in the post-test that made many senior
paramedics overtriage.

\hypertarget{limitations}{%
\section{Limitations}\label{limitations}}

Our study has several limitations. First, the number of paramedics was
limited. Second, the seniority and major trauma experience among the EMT
intermediates were not recorded. Hence, the association between the
seniority of EMT intermediates and its training effect cannot be
evaluated. Third, training course impact was evaluated based on the
tests, not clinical performance. Finally, the cut point of 12 years (in
EMT seniority) and 5 years (in paramedic year) was not based on medical
evidence but for the middle point of the data distribution to enhance
the comparability. Actually, there was scarce evidence about the
association between the EMT seniority and their trauma triage
performance; therefore, we just choice the median year as the cut point.
Further studies should focus on the EMT seniority and their clinical
performance.

\hypertarget{conclusions}{%
\section{Conclusions}\label{conclusions}}

The training course improves the field triage accuracy in Taipei City.
Anatomical FTS aspects are more difficult to improve by a single
training program, especially in senior paramedics. Future teaching
programs should emphasize FTS anatomy aspects.

\hypertarget{acknowledgments}{%
\section{Acknowledgments}\label{acknowledgments}}

We appreciate the excellent performance of EMTs and the quality
assurance of the Ambulance Division of Taipei City. The authors would
also like to express their thanks to the staff of National Taiwan
University Hospital-Statistical Consulting Unit (NTUH-SCU) for
statistical consultation and analyses.

\hypertarget{conflicts-of-interest-statement}{%
\section{Conflicts of Interest
Statement}\label{conflicts-of-interest-statement}}

None.

\hypertarget{funding}{%
\section{Funding}\label{funding}}

This study was funded by the Taiwan Ministry of Science and Technology
(MOST 105-2314-B-002- 200-MY3 and MOST 105-2314-B-002-182).

\hypertarget{reference}{%
\section*{Reference}\label{reference}}
\addcontentsline{toc}{section}{Reference}

\hypertarget{refs}{}
\begin{CSLReferences}{1}{0}
\leavevmode\vadjust pre{\hypertarget{ref-1}{}}%
McCoy, C. E., Chakravarthy, B., \& S, L. (2013). Guidelines for field
triage of injured patients: In conjunction with the morbidity and
mortality weekly report published by the center for disease control and
prevention. \emph{West J Emerg Med}, \emph{14}, 69--76.
\url{https://doi.org/10.5811/westjem.2013.1.15981}

\leavevmode\vadjust pre{\hypertarget{ref-2}{}}%
MacKenzie, E. J., Rivara, F. P., \& Jurkovich, G. J. (2006). A national
evaluation of the eff ect of trauma-center care on mortality. \emph{N
Engl J Med}, \emph{354}, 366--378.
\url{https://doi.org/10.1056/NEJMsa052049}

\leavevmode\vadjust pre{\hypertarget{ref-3}{}}%
Chen, C. H., Shin, S. D., \& Sun, J. T. (2020). Association between
prehospital time and outcome of trauma patients in 4 asian countries: A
cross-national, multicenter cohort study. \emph{PLoS Med},
\emph{17:e1003360}. \url{https://doi.org/10.1371/journal.pmed.1003360}

\leavevmode\vadjust pre{\hypertarget{ref-4}{}}%
Sasser, S. M., Hunt, R. C., \& Faul, M. (2012). Guidelines for field
triage of injured patients: Recommendations of the national expert panel
on field triage, 2011. \emph{MMWR Recomm Rep}, \emph{61}(RR-1), 1--20.

\leavevmode\vadjust pre{\hypertarget{ref-5}{}}%
Newgard, C. D., Kampp, M., \& Nelson, M. (2012). Deciphering the use and
predictive value of {``emergency medical services provider judgment''}
in out-of-hospital trauma triage: A multisite, mixed methods assessment.
\emph{J Trauma Acute Care Surg}, \emph{72}, 1239--1248.
\url{https://doi.org/10.1097/TA.0b013e3182468b51}

\leavevmode\vadjust pre{\hypertarget{ref-6}{}}%
Newgard, C. D., Zive, D., \& Holmes, J. F. (2011). A multisite
assessment of the american college of surgeons committee on trauma field
triage decision scheme for identifying seriously injured children and
adults. \emph{J Am Coll Surg}, \emph{213}, 709--721.
\url{https://doi.org/10.1016/j.jamcollsurg.2011.09.012}

\end{CSLReferences}
\end{multicols}


%\layout
\end{document}
