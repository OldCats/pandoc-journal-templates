% Options for packages loaded elsewhere
\PassOptionsToPackage{unicode}{hyperref}
\PassOptionsToPackage{hyphens}{url}
%
\documentclass[
  20pt
]{article}

\usepackage{layout}
\usepackage{authblk}
\usepackage[symbol,flushmargin]{footmisc}
\newcommand{\pcdoc}{Pandoc-crossref documentation}
\renewcommand{\thefootnote}{\fnsymbol{footnote}}
\renewcommand{\footnoterule}{%
  \kern 2em  % footnote 水平線上空間
  \hrule width \textwidth height 0.5pt % footnote 水平線寬
  \kern 0.5em % footnote 水平線下空間
}

\usepackage{ifthen}
% \usepackage[256, showframe]{geometry}
\usepackage[top=2.5cm, bottom=2.5cm]{geometry}
\usepackage{amsmath,amssymb}
\usepackage{multicol}
\setlength{\columnsep}{1.2cm}
\usepackage{graphicx}
\usepackage{lmodern}
\usepackage{iftex}

\usepackage{indentfirst}
\usepackage{fancyhdr}
\setlength{\parskip}{1em}
\usepackage{titlesec}

% \titlespacing{command}{left spacing}{before spacing}{after spacing}[right]
\titlespacing\section{0pt}{1pt}{0pt plus 2pt minus 2pt}
\titlespacing\subsection{0pt}{1pt plus 4pt minus 2pt}{0pt plus 2pt minus 2pt}
\titlespacing\subsubsection{0pt}{1pt plus 4pt minus 2pt}{0pt plus 2pt minus 2pt}

\geometry{a4paper, scale=0.75}

%------------header/footer setting-----------%

\pagestyle{fancy}{}
\fancyhf{}
\setcounter{page}{22}

\fancyhead[L] % 頁首左邊
{\ifthenelse{\value{page}=22} % 第1頁
    {\footnotesize \textbf{Journal of Acute Medicine} 11(1): 22-27, 2021\\[0.2em]
    \textbf{DOI}:10.6705/j.jacme.202103\_11(1).0004\\[0.2em]
    \textbf{Brief Report}
    }
    {\ifthenelse{\isodd{\value{page}}}
        {} % 其他奇數頁
        {Chiu et al.}} % 其他偶數頁
}

\fancyhead[R] % 頁首右邊
{\ifthenelse{\value{page}=22} % 第1頁
    {\includegraphics[width=1.5cm]{ap.jpg}
     \includegraphics[width=1.5cm]{jacme.jpg}
     \includegraphics[width=1.5cm]{doi-11-1-04.jpg}}
    {\ifthenelse{\isodd{\value{page}}}
        {\footnotesize Trauma Field Triage Training of EMT} % 其他奇數頁
        {}} % 其他偶數頁
}

\fancyfoot[L] % 左邊頁尾
{\ifthenelse{\isodd{\value{page}}}
    {}
    {\thepage\hspace{5pt}\footnotesize Journal of Acute Medicine 11(1) 2021}}

\fancyfoot[R] % 右邊頁尾
{\ifthenelse{\isodd{\value{page}}}
    {{\footnotesize Journal of Acute Medicine 11(1) 2021}\hspace{5pt}\thepage}
    {}}


\renewcommand{\headrulewidth}{0pt}

%-------------------------------------------%


%------------footnote without marker--------%
\newcommand\blfootnote[1]{%
  \begingroup
  \renewcommand\thefootnote{}\footnote{#1}%
  \addtocounter{footnote}{-1}%
  \endgroup
}
%-------------------------------------------%

% luatex or xetex setting
  \usepackage{unicode-math}
  \defaultfontfeatures{Scale=MatchLowercase}
  \defaultfontfeatures[\rmfamily]{Ligatures=TeX,Scale=1}


% \makeatletter

% \setlength{\parskip}{6pt plus 2pt minus 1pt}}
\usepackage{hyperref} %for citing reference

% \urlstyle{same} % disable monospaced font for URLs
\usepackage{longtable,booktabs,array}
\usepackage{calc} % for calculating minipage widths
% Correct order of tables after \paragraph or \subparagraph
\usepackage{etoolbox}
\makeatletter
\patchcmd\longtable{\par}{\if@noskipsec\mbox{}\fi\par}{}{}
\makeatother
% Allow footnotes in longtable head/foot
\IfFileExists{footnotehyper.sty}{\usepackage{footnotehyper}}{\usepackage{footnote}}
\makesavenoteenv{longtable}
\setlength{\emergencystretch}{3em} % prevent overfull lines
\providecommand{\tightlist}{%
  \setlength{\itemsep}{0pt}\setlength{\parskip}{0pt}}
\setcounter{secnumdepth}{-\maxdimen} % remove section numbering
\ifLuaTeX
  \usepackage{selnolig}  % disable illegal ligatures
\fi
\newlength{\cslhangindent}
\setlength{\cslhangindent}{1.5em}
\newlength{\csllabelwidth}
\setlength{\csllabelwidth}{3em}
\newenvironment{CSLReferences}[2] % #1 hanging-ident, #2 entry spacing
 {% don't indent paragraphs
  \setlength{\parindent}{0pt}
  % turn on hanging indent if param 1 is 1
  \ifodd #1 \everypar{\setlength{\hangindent}{\cslhangindent}}\ignorespaces\fi
  % set entry spacing
  \ifnum #2 > 0
  \setlength{\parskip}{#2\baselineskip}
  \fi
 }%
 {}
\usepackage{calc}
\newcommand{\CSLBlock}[1]{#1\hfill\break}
\newcommand{\CSLLeftMargin}[1]{\parbox[t]{\csllabelwidth}{#1}}
\newcommand{\CSLRightInline}[1]{\parbox[t]{\linewidth - \csllabelwidth}{#1}\break}
\newcommand{\CSLIndent}[1]{\hspace{\cslhangindent}#1}




\makeatletter 

% maketitle and author
\renewcommand{\maketitle}{\bgroup\setlength{\parindent}{0pt}
\begin{flushleft}
    \begin{minipage}{14cm}
        \textbf{\Large\@title}
    \end{minipage}

    \vspace*{3pt}
        \@author
\end{flushleft}\egroup
}



\renewcommand*{\Authfont}{\raggedright\sffamily\small}
\renewcommand*{\Affilfont}{\raggedright\rmfamily\scriptsize} % 修改機構名稱的字體與大小
\renewcommand\Authands{, } % 去掉 and 前的逗號
\date{} % 去掉日期





\begin{document}

\footnotetext[0]{Received: May 15, 2019; Revised: January 6, 2020 (3rd); Accepted: February 11, 2020}

$if(title)$
\title{$title$}
$endif$

$if(author)$
  $for(surname)$
    \author[2]{$surname$ $given-names$}
  $endfor$
$endif$

$for(author)$\author[$for(author.affiliation)$$author.affiliation$$sep$,$endfor$$if(author.corresp)$,\dagger$endif$]{$author.given-names$ $author.surname$$if(author.footnote)$\footnote[0]{\textsuperscript{\ast}$author.footnote$}$endif$}$endfor$

\affil[1]{Department of Emergency Medicine,
        Far Eastern Memorial Hospital, 
        New Taipei City, 
        Taiwan}
\affil[2]{Department of Emergency Medicine, 
        National Taiwan University Hospital, 
        Taipei, 
        Taiwan}
\affil[3]{National Fire Agency, 
        Ministry of the Interior, 
          Taiwan}
\affil[4]{Taipei City Fire Department, 
        Taipei, 
        Taiwan}
\affil[5]{Department of Emergency Medicine, 
        National Taiwan University Hospital Yun-Lin Branch, 
        Yunlin, 
        Taiwan}
        

\vspace*{5ex}

% \blfootnote{\textsuperscript{\dagger}These authors contributed equally to this work.}   

\maketitle
\thispagestyle{fancy} % render Title, author and affiliation

\vspace{1ex}

% Abstracts and keywords
\begin{minipage}{15cm}
    \vspace{5ex}
    



    {\parindent 2em Injury is a leading cause of death among young adults. An accurately
implemented filed triage scheme (FTS) by emergency medical technicians
(EMTs) is the first step for delivering right patients to the right
hospital. However, the training effect of FTS on EMTs with different
levels and backgrounds has scarcely been reported. We evaluated training
effects of FTS among EMTs in Taipei. Standard FTS contains physiologic
status, anatomical sites of injury, and mechanism of injury criteria.
The intervention was a 30-minute lecture and pre-and-post tests, each
containing five questions about trauma severity judgment (i.e.,
mechanism of injury {[}2 questions{]}, anatomic sites of injury {[}2
questions{]}, and physiological status {[}1 question{]}). The change in
EMT accuracy was measured before and after training. Subgroup analyses
were performed across EMTs with different levels and seniorities. From
September 1, 2015 to March 31, 2016, 821 EMTs were enrolled, including
740 EMT-intermediates and 81 paramedics. Overall, EMT accuracy improved
after the intervention in the intermediate (73.2\% vs.~85.5\%, p
\textless{} 0.05) and paramedic (76.0\% vs.~85.7\%, p \textless{} 0.01)
groups. All trainees showed improvements in physiology and mechanism
criteria, but paramedics showed decreased accuracy in anatomic criteria.
The subgroup analysis showed that accuracy positively associated with
prehospital care experience for major trauma cases 1 year before the
training course, and the anatomical criterion accuracy was adversely
associated with paramedic seniority. Field triage training can improve
EMT accuracy for FTS. The anatomical aspect is more diffi cult to
improve and should be emphasized in FTS training courses.}

    \vspace{1em}
    \textbf{Key words:} \textit{emergency medical service (EMS), emergency
medical technician (EMT), education and training, filed triage scheme
(FTS), trauma}

\end{minipage}

\vspace{2em}

\setlength{\parindent}{2em}


$body$

%\layout
\end{document}
